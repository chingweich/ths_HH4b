\chapter{Final Result} \label{chap:6}

The final result based on the 95$\% $ CLs of upper limit on signal cross section $\times $ branch ratio (pp $\rightarrow$ X $\rightarrow$ HH $\rightarrow b\bar{b}b\bar{b}$) is shown in the chpter.

\section{Asymptotic CLs} 
The analysis uses CLs to find the upper limit of cross section\citep{0954-3899-28-10-313,NicolasChanon}. The CLs method tests whether the data is more consistent with a signal + background hypothesis or with a background-only hypothesis when the statistic of data is low. The CLs method starts with calculating likelood ration of the two hypotheses. 
\begin{equation} \label{eq1}
\begin{split}
Q = \frac{\textbf{\textsl{L}}(\textsl{N}_{data},\textsl{N}_{S}+\textsl{N}_{B})}{\textbf{\textsl{L}}(\textsl{N}_{data},\textsl{N}_{B})}\\
\textbf{\textsl{L}}(n,x) = \frac{e^{-x}}{n!}x^n
\end{split}
\end{equation}
Then, the prabability density function on -2$\textit{ln}$(Q) of the two hypotheses are generated by toy-experiments using $\textsl{N}_{S} + \textsl{N}+{B}$ and $\textsl{N}+{B}$ as $\textsl{N}_{data}$ in the equation separately. The $CL_{s+b}$ represents the prabability of results less consisentent with a signal + background hypothesis, while the $CL_{b}$ represents the prabability of results less consisentent with a background-only hypothesis.
\begin{equation} \label{eq1}
\begin{split}
CL_{s+b} = P_{s+b}(lnQ \leq lnQ_{obs}) = \int^{lnQ_{obs}}_{-\infty} \frac{dP_{s+b}}{dlnQ} dlnQ \\
CL_{b} = P_{b}(lnQ \leq lnQ_{obs}) = \int^{lnQ_{obs}}_{-\infty} \frac{dP_{b}}{dlnQ} dlnQ
\end{split}
\end{equation}
The CLs defined as 
\begin{equation} \label{eq1}
\begin{split}
CL_{s} = \frac{CL_{s+b}}{CL_{b}}
\end{split}
\end{equation}
can be approximately seen as signal-only hypothesis. The 95$\% $ confidence level of CLs means that the prabability to observed the data excessing the signal + background hypothesis which is normalized to the prabability to observed the data excessing the background-only hypothesis is less thna 5 $\% $. The observed  exclusion of 95$\% $ confidence level of CLs is calculated by integrated to a given $lnQ_{obs}$ where CLs $<$ 0.05. The expected exclusion limit is done in same procedure where $lnQ_{obs}$ is replaced by $lnQ_{b}$ in the equation. The Asymptotic CLs is obtained by CMS Higgs combination tool\citep{higgsCombine}.

\section{95$\% $ upper limit}
 